\capitulo{6}{Trabajos relacionados}

Para comenzar a trabajar en el desarrollo de mi aplicación tuve que buscar información relacionada con la aplicación que quería llevar a cabo.

Para ello, he realizado una investigación a través de la Web sobre  aplicaciones que tengan una función similar a la mía. 

Aunque no existe una gran variedad de programas que realicen tareas similares a $PLGRAM$, entre los dos programas recomendados por mi tutor he encontrado información suficiente para poder llevar a cabo mi aplicación. Dichos programas son $PLQUIZ$ y $BURGRAM$.

Dado que ambos programas han sido desarrollados por alumnos de la Universidad de Burgos, como trabajos de final de grado, mi tutor pudo facilitarme su estructura y toda la documentación que estimó necesaria para la correcta ejecución de $ PLGRAM $.


\section{PLQUIZ}

PLQUIZ\footnote{Roberto Izquierdo Amo, Julio 2014, Universidad de Burgos} es una herramienta de escritorio que permite generar preguntas de test aleatorias (tipo quiz, cloze, de texto libre...) sobre problemas de algoritmos de análisis léxico. El formato utilizado para generarlas preguntas es .XML para importarse a entornos virtuales de aprendizaje (Moodle), y .TEX para  su impresión en papel a través de la obtención de código \LaTeX{}.

He seguido sus pasos en el formato de la interfaz gráfica, pues me parece que está muy bien organizada y muy clara para el usuario, permitiendo acceder rápidamente a las opciones disponibles.

\imagen{PLQUIZ}{PLQUIZ}


\section{BURGRAM}

BURGRAM\footnote{Carlos Gómez Palacios, Enero 2008, Universidad de Burgos} es una herramienta que permite seguir paso a paso el proceso de los distintos modos de análisis sintáctico (tanto ascendente como descendente). La aplicación permite editar gramáticas, visualizar el proceso completo de operación del algoritmo sobre las mismas y la generación de informes con los resultados.

He reutilizado varias de las clases de este programa pues realizan el análisis que necesitaba para obtener los datos que posteriormente he tratado para obtener los ficheros $.xml$ 

\imagen{BURGRAM}{BURGRAM}

\section{SEFALAS}

SEFALAS es una herramienta que muestra el transcurso de la fase de análisis léxico y de analisis sintáctico. Es de bastante utilidad dado que existen multitud de estrategias de análisis sintáctico y, por regla general, puede provocar confusión.
\imagen{SEFALAS}{SEFALAS}


