\apendice{Documentación de usuario}

\section{Introducción}
En esta sección se explica cómo llevar a cabo la instalación de las herramientas implementadas en el presente proyecto y los requisitos del sistema.
\section{Requisitos de usuarios}
Los requisitos mínimos necesarios son los siguientes:
\subsection{Requisitos hardware}
Requisitos mínimos:
\begin{itemize}
\item Intel core i3.
\item 128 MB  de memoria RAM.
\item Resolución de pantalla igual o superior a 1152 x 864, solo para la GUI.
\end{itemize}

Requisitos recomendados:

\begin{itemize}

\item Intel core i7.
\item 1 GB de memoria RAM.
\item Resolución de pantalla igual o superior a 1280 x 960 , solo para la GUI.
\end{itemize}

\subsection{Requisitos software}

Los requisitos software necesarios son:
\begin{itemize}
\item Sistema operativo $Windows$ o $Linux$.
\item Máquina virtual de java ($JDK 8$).
\item Consola de sistema operativo capaz de interpretar Unicode, solo para la interfaz en línea de comandos. Si no soportara Unicode, las tildes u otros símbolos podrían ser mal representados.
\end{itemize}
\section{Instalación}
La aplicación se distribuye de dos formas diferentes: $PLGRAM$ y $PLGRAMlineCommand$:

\begin{itemize}
\item Para ejecutar $PLGRAM$ se dispone de un fichero jar ejecutable.
\imagen{iconoPLGRAM}{Icono PLGRAM}
\item Para ejecutar $PLGRAMlineCommand$ se debe acceder mediante la línea de comandos a la carpeta que contiene el fichero.

Esta carpeta debe contener también una carpeta llamada \textit{gramáticas} donde estarán las gramáticas guardadas. Además, también estarán en ese directorio las plantillas necesarias.

\imagen{inicioconsola}{Acceso a la carpeta contenedora de $PLGRAM$ desde la consola}
\end{itemize}


\section{Manual del usuario}

\subsection{Aplicación línea de comandos}

Para ejecutar la aplicación se debe escribir el siguiente comando:
\begin{verbatim}
 java -jar PLGRAMlineCommand.jar -g -t [-i] [-ca] [-o] 
\end{verbatim}

 
 Los diferentes argumentos son:
 \begin{itemize}
 	\item $-g$ Nombre de la gramática a analizar.
 	\item $-t$ Tipo de análisis a realizar $[LL,LR,SLR,LALR].$
 	\item $-i$ Extensión del informe $[TEX, XML, ALL]$.
 	\item $-ca$ Cadena a analizar.
 	\item $-o$ Nombre del fichero de salida.
 	
\end{itemize} 



La gramática $-g$ y el tipo de análisis a realizar $-t$ son obligatorios. La gramática debe estar en la carpeta gramáticas.
 
El parámetro $-t$ tiene diferentes opciones:
  \begin{itemize}
 	 \item $LL$ Para realizar un análisis $LL$.
 	 \item $LR$ Para realizar un análisis $LR$.
 	 \item $SLR$ Para realizar un análisis $SLR$.
 	 \item $LALR$ Para realizar un análisis $LALR$.
 \end{itemize}
  
 
Si no se introduce ningún parámetro se indicará que falta el parámetro de la gramática, ya que como se ha comentado anteriormente, es un parámetro obligatorio.
 
Así, si se introduce la siguiente línea de código 
\begin{verbatim}
 java -jar PLGRAMlineCommand.jar  
\end{verbatim}

La aplicación requerirá la gramática, tal y como se muestra a continuación:

\imagen{consola2}{$PLGRAM$ sin argumento gramática}
 
La misma situación se presenta con el parámetro tipo de análisis a realizar.
 
Si se  introduce la línea de código
\begin{verbatim}
java -jar PLGRAMlineCommand.jar -g gramatica1.yc 
\end{verbatim}
en la que no se ha especificado el tipo de análisis, la aplicación lo solicita, como se muestra a continuación:
\imagen{consola3}{$PLGRAM$ sin argumento tipo de análisis}

Si se introducen los parámetros de la gramática y del tipo de análisis que se quiere realizar, se ejecuta el programa con el resto de valores por defecto:
\begin{itemize}
\item $-i$ con valor $ALL$.
\item $-o$ que se denominará de manera genérica como el tipo de análisis elegido concatenado al nombre de la gramática seleccionada. Por ejemplo, para un análisis $LL$ y una gramática $G3$, los ficheros generados se denominarán $LLG3$, cada uno con su extensión correspondiente.
\item $-ca$ no se generará ninguna tabla de primeros pasos de la traza de análisis.
\end{itemize}


En el supuesto de que se quiera modificar alguno de los valores que por defecto se les asignan a los parámetros anteriores, bastará con introducir en cada uno de ellos los argumentos deseados:
\begin{itemize}
\item $-i$ se podrá elegir la extensión del fichero de salida, que será $XML$ en el caso de querer obtener un cuestionario virtual o  $TEX$ en el caso de que se quiera para una futura impresión.
\item $-o$ se podrá nombrar los ficheros de salida según se desee.
\item $-ca$ se podrá introducir una cadena para  que la aplicación genere una tabla de primeros pasos de la traza de análisis de dicha cadena.
La cadena deberá introducirse entrecomillado y cada item separado por un espacio. Por ejemplo, $-ca$ "$item1$ $item2$ $item3$ $item4$".
\end{itemize}


Una vez introducidos todos los parámetros, se muestra la información de la gramática elegida a través de la consola.

A continuación se muestran los datos obtenidos al introducir una gramática $gramatica1.yc$ y un tipo de análisis $LL$. El resto de parámetros se han dejado por defecto.

\imagen{consola4}{$PLGRAM$ Con argumentos gramática y tipo de análisis}

Como se puede observar, al final de la información obtenida, la consola nos muestra que los ficheros se han generado correctamente y su ruta.


\subsection{Aplicación interfaz gráfica}
FALTA


\subsection{Ficheros obtenidos}
\subsubsection{Fichero .XML}

El fichero $XML$ obtenido tiene una estructura general común sea cual sea la gramática y el tipo de análisis realizado. Esta estructura sigue el siguiente esquema general:
\imagen{EstructuraXML}{Estructura del fichero.XML}

Para poder utilizar el fichero.XML obtenido es necesario poder acceder a la plataforma de aprendizaje de moodle.

Los pasos que se deben seguir son:
\begin{enumerate}
\item Ir a la página principal del curso.
\item Hacer clic en Banco de preguntas desde el bloque de Administración.
\item Pulsar en Importar archivos.
\item Seleccionar la opción de formato XML Moodle.
\item Desde la siguiente ventana pulsar en Examinar.
\item Seleccionar el fichero.XML en nuestro ordenador.
\item Pulsar en Importar este archivo.
\end{enumerate}

Los puntos 5 y 6 se pueden realizar arrastrando el archivo al recuadro que se muestra.
\imagen{Moodle1}{Importar fichero.XML en moodle}

A la hora de importar el fichero.XML hay que tener en cuenta lo siguiente:
\begin{itemize}
\item Los nombres de los archivos y carpetas no deben tener caracteres especiales como palabras acentuadas, tabulaciones, ñ, retornos de carro, espacios en blanco, ni símbolos de sistema como / o : , además no es recomendable combinar mayúsculas y minúsculas en los nombres.
\item El tamaño del archivo no debe superar el límite permitido.

\end{itemize} 

Una vez importado el fichero.XML deseado, aparece una primera previsualización de los datos que se obtendrán a continuación. Basta con dar a continuar para acceder a la siguiente pantalla.

En ella, aparecen todos los ficheros que se hayan importado. Para visualizar uno de ellos, se hace click en la lupa del archivo correspondiente.
\imagen{moodle2}{Lista de ficheros.XML en moodle}

Un ejemplo de cuestionario obtenido sería el siguiente:
\imagen{moodle3}{Ejemplo de cuestionario resuelto en moodle }

\subsubsection{Fichero .TEX}

Una vez realizado el cuestionario siguiendo cualquiera de las vías explicadas anteriormente, se obtiene un fichero .TEX en una carpeta llamada \textit{informes} situada en el directorio donde se ejecuta el ejecutable .jar.

Al tratarse de un fichero .TEX se requiere un editor de textos adaptado a \LaTeX{} para utilizarlo.

Una vez abierto el fichero, su tratamiento es muy sencillo. Bastará con compilar el código y se obtendrá directamente un visionado del cuestionario que se ha realizado.

Se muestra un ejemplo de un fichero .TEX abierto con \TeX{}MaKer. De manera general, se puede observar el código a la izquierda y del documento generado a la derecha.

\imagen{ejemplotex1}{Ejemplo fichero .TEX}

Para una mejor comprensión, se muestra un detalle de cómo compilar el código para obtener el documento. Para ello, basta con hacer click en las flechas resaltadas en la parte superior de la pantalla del editor.

\imagen{ejemplotex2}{Compilación de un fichero .TEX}

Es necesario destacar que para cada tipo de gramática elegida, se genera un fichero .TEX diferente siguiendo los requisitos de cada una de ellas. 

La aplicación ofrece dos opciones en lo que a los ficheros .TEX respecta. Así, se podrá obtener un documento en el que el cuestionario esté completo con las respuestas correctas y otro documento en el que el cuestionario se encuentre vacío, pensado para ser respondido por un tercero.

\begin{itemize}
\item Obtención del cuestionario completo. Es la opción que se genera por defecto y por tanto no será necesario tratar el código para su obtención. De cualquier manera, debería estar activa la siguiente línea:
\begin{verbatim}
\newcommand{\h}[1]{#1}
\end{verbatim}
\imagen{ejemplotex3}{Obtención del cuestionario completo}
\item Obtención del cuestionario en blanco. Para ello, bastaría con eliminar el símbolo $\%$ de delante de la siguiente línea:

\begin{verbatim}
%\renewcommand{\h}[1]
\end{verbatim}

\imagen{ejemplotex4}{Obtención del cuestionario en blanco: Descomentar}
Así, ambas líneas deberían estar activas.
\imagen{ejemplotex5}{Obtención del cuestionario en blanco}

\end{itemize}

Siempre que se compile código, se crea un archivo .pdf en el mismo directorio donde se encuentra el fichero .TEX, con los cuestionarios elegidos.