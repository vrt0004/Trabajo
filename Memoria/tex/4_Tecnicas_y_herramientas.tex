\capitulo{4}{Técnicas y herramientas}

\section{Técnicas}

\subsection{SCRUM}
SCRUM\cite{scrum} es una metodología ágil en la que se aplican de manera regular un conjunto de buenas prácticas para trabajar colaborativamente y obtener el mejor resultado posible de un proyecto.

Se basa en la creación de procesos iterativos e incrementales, así se pueden realizar estimaciones de rendimiento futuro sobre las cuales se pueden tomar decisiones y controlar los riesgos.
\subsection{UML\cite{UML}}
Lenguaje Unificado de Modelado (Unified Modeling Language)\footnote{\url{http://www.uml.org/}} es el lenguaje de modelado de sistemas de software más conocido y utilizado en la actualidad en el diseño de aplicaciones Orientadas a Objetos.

Es un lenguaje gráfico para visualizar, especificar, construir y documentar un sistema.

Es importante remarcar que UML es un lenguaje de modelado para especificar o para describir métodos o procesos. Se utiliza para definir un sistema, para detallar los artefactos en el sistema y para documentar y construir. En otras palabras, es el lenguaje en el que está descrito el modelo.
\section{Herramientas}

Todas las herramientas utilizadas para la elaboración del proyecto son gratuitas o se pueden obtener versiones de prueba en Internet.

\subsection{\textit{Java}}
\textit{Java}\footnote{\url{https://www.java.com.es}} es un lenguaje de programación de propósito general, concurrente y orientado a objetos.

Se trata de un lenguaje fuerte y estáticamente tipado.

Distingue entre errores en tiempo de compilación y errores en tiempo de ejecución.

\textit{Java} es un lenguaje de alto nivel, en el sentido de que los detalles de representación no son accesibles al programador. 
Incluye una administración automática del almacenamiento de datos en forma del recolector de basura, diseñado para evitar los problemas relacionados con la liberación manual de memoria. Es uno de los lenguajes de programación mas usados en el mundo.\cite{java}

\textit{Java SE 8} es la edición más reciente del lenguaje de programación \textit{Java}, y representa la mayor evolución del mismo en su historia.

Incluye nuevas características, mejoras y correcciones de bugs para mejorar la eficacia en el desarrollo y la ejecución de programas \textit{Java}. 
 
Las librerías de la plataforma \textit{Java} mantienen una evolución paralela a la del lenguaje.

\subsection{\textit{Eclipse}}
\textit{Eclipse}\footnote{\url{https://eclipse.org/}} es una plataforma de software compuesto por un conjunto de herramientas de programación de código abierto multiplataforma para el desarrollo integrado (\textit{IDE}) de aplicaciones \textit{Java}  y una base para productos basados en \textit{Eclipse Platform}.


\subsection{\textit{JUnit 4}}
\textit{JUnit}\footnote{\url{http://junit.org/}} es un sistema software utilizado para realizar pruebas sobre código \textit{Java}, formando parte de la familia de herramientas de pruebas \textit{xUnit}.
Es una de las librerías \textit{Java} más utilizadas en proyectos de código abierto.

La versión 4 de \textit{JUnit} extiende y simplifica la funcionalidad de anteriores versiones, haciendo un uso extensivo del sistema de anotaciones de \textit{Java}.


\subsection{\textit{Apache Maven}}
\textit{Maven}\footnote{\url{http://maven.apache.org/}} es una herramienta de administración y construcción de proyectos usada principalmente con \textit{Java}.

Su funcionamiento se basa en la existencia de un fichero de configuración XML, el \textit{POM} (Project Object Model).
Este fichero define la construcción, emisión, documentación, empaquetado, pruebas, manejo de dependencias y múltiples otras tareas de manera centralizada.

\subsection{\textit{JavaCC}}
\textit{JavaCC}\footnote{\url{https://javacc.java.net/}} (\emph{Java Compiler Compiler}) es un generador de analizadores sintácticos \textit{Java}.
Funciona mediante la especificación de una gramática en un formato propio.
A partir de este fichero, la herramienta genera un programa Java capaz de procesar texto y reconocer coincidencias con la gramática.

Además del generador en sí, \textit{JavaCC} proporciona una serie de herramientas relacionadas, como un constructor de árboles (\textit{JJTree}) y un informe en caso de error de los mejores entre los analizadores sintácticos disponibles.

\subsection{\textit{VersionOne}}
\textit{VersionOne}\footnote{\url{https://www.versionone.com/}} es un sistema de control de versiones, diseñado para trabajar con proyectos de cualquier tamaño.
\textit{VersionOne} permite mantener múltiples ramas locales independientes, cuya creación, modificación y combinado resulta poco costoso.
Esto permite aislar tareas, trabajando e incorporando cada una de manera totalmente separada.

\subsection{\textit{Github}}
\textit{Github}\footnote{\url{https://github.com/}} proporciona un servicio de almacenamiento de repositorios remotos y un entorno de colaboración para desarrolladores.

\textit{Github} proporciona herramientas de seguimiento de proyectos, incluyendo una wiki y un sistema de seguimiento de problemas (\emph{issue tracker}) por repositorio.
También es compatible con otras aplicaciones web, como por ejemplo \textit{Pivotal Tracker}, que facilitan su integración en el proceso de desarrollo.

Enlace al repositorio de este proyecto:  \url{https://github.com/vrt0004/Trabajo}

\subsection{\textit{Moodle}}
\textit{Moodle} (\emph{Modular Object-Oriented Dynamic Learning Environment}) es una plataforma de enseñanza virtual (\emph{e-learning}) desarrollada como software libre.

Es una aplicación para crear y gestionar plataformas educativas, espacios donde un centro educativo, institución o empresa, gestiona recursos proporcionados por sus docentes, organiza el acceso a los mismos y permite la comunicación entre todos los implicados (alumnado y profesorado).

Una de las características de \textit{Moodle} es su capacidad para importar cuestionarios a partir de varios formatos, incluyendo texto plano, el formato propietario \textit{Gift} y a partir de documentos XML.

\subsection{\LaTeX{}}
\LaTeX{}\footnote{\url{https://latex-project.org/}} es un lenguaje de marcadores para la preparación de documentos comúnmente usado en publicaciones técnicas o científicas.

Es un procesador de texto «\emph{What you see is not what you get}», lo cual significa que lo que vemos durante la edición no es el documento final, sino las instrucciones para generarlo.
Esto permite separar casi completamente el contenido del documento de su formato. La salida obtenida será la misma con independencia del dispositivo o sistema operativo empleado para su visualización o impresión.

\subsection{\TeX{}Maker}
\TeX{}Maker\footnote{\url{http://www.xm1math.net/texmaker/}} es un editor libre de /\LaTeX{}, moderno y multiplataforma para Windows, Linux y MacOSX que integra las herramientas necesarias para desarrollar documentos con \LaTeX{}, en una sola aplicación.

\TeX{}Maker incluye soporte \textit{Unicode}, la corrección ortográfica, auto-completado, plegado de código y un visor de pdf con el apoyo \textit{SyncTeX} y el modo de visualización continua. \TeX{}Maker es fácil de usar y configurar. Se distribuye bajo la licencia \textit{GPL}.

\subsection{MiK\TeX{}}
MiK\TeX{} es una distribución \TeX{}/\LaTeX{} libre de código abierto para \textit{Windows}.
Una de sus características es la capacidad que tiene para instalar paquetes automáticamente sin necesidad de intervención del usuario. Al contrario que otras distribuciones, su instalación es muy sencilla.

\subsection{\textit{Mustache}}
\textit{Mustache}\footnote{\url{http://mustache.github.io/mustache.1.html}} es un sistema de plantillas de lógica descendente para HTML, XML y otros muchos.

Se llama de lógica descendente porque no tiene comandos if, else, ni bucles for. En cambio, sólo se trabaja con etiquetas.
Algunas etiquetas son reemplazadas por un valor, por un conjunto de valores o por nada.

\subsection{\textit{XML}}
\emph{XML} (\emph{Extensive Markup Language}) es un lenguaje de etiquetas que se utiliza para crear documentos estructurados, compuestos de entidades que contienen en su interior datos u otras entidades.
El estándar fue producido y es desarrollado por el \emph{World Wide Web Consortium}\footnote{\url{http://www.w3.org/}}.

Podemos verificar que un documento \emph{XML} tiene el formato correcto validándolo contra un lenguaje de definición de esquemas: \textit{DTD}\footnote{\url{http://es.wikipedia.org}},(\emph{Document Type Definition}), \emph{XML Schema}\footnote{\url{http://www.w3.org/TR/xmlschema-0/}}, \textit{RELAX NG}\footnote{\url{http://www.relaxng.org/}}, etc.

También podemos verificar que un documento está <<bien formado>>, es decir, que cumple una serie de reglas gramaticales mínimas.

\subsection{\textit{Commons CLI}}
La biblioteca \textit{Apache Commons CLI}\footnote{\url{http://commons.apache.org/proper/commons-cli/}} proporciona una API para analizar las opciones de línea de comandos pasados a los programas.

También es capaz de imprimir mensajes de ayuda que detallan las opciones disponibles para una herramienta de línea de comandos.

\subsection{\textit{ObjectAid UML Explorer}}
\textit{ObjectAid UML Explorer}\footnote{\url{http://www.objectaid.com/}} es un complemento de visualización de código para \textit{Eclipse}.
Permite mostrar el código fuente de un proyecto Java en forma de diagramas \textit{UML}, reflejando el estado y las relaciones en el código, y actualizándose a medida que el código cambia.


\subsection{\textit{EclEmma}}
\textit{EclEmma}\footnote{\url{http://www.eclemma.org/}} es una herramienta que permite examinar la cobertura de pruebas en un proyecto \textit{Java}.
Dispone un un complemento para \textit{Eclipse} que permite realizar las comprobaciones directamente desde el \textit{IDE}.

\subsection{\textit{Jabref}}
\textit{Jabref}\footnote{\url{http://www.jabref.org/}} es un editor de referencias bibliográficas que permite introducir las referencias bibliográficas en \TeX{}Maker de forma sencilla.

\subsection{\textit{SourceMonitor}}
\textit{SourceMonitor}\footnote{\url{http://www.campwoodsw.com/sourcemonitor.html}} es una herramienta que permite obtener las métricas del codigo de la aplicación \textit{PLGRAM}.

\subsection{\textit{DIA}}
\textit{DIA}\footnote{\url{https://sourceforge.net/projects/dia-installer/}} es una herramienta para dibujar diagramas de estructuras. \textit{Dia Diagram Editor} es un software gratuito de dibujo de código abierto.




