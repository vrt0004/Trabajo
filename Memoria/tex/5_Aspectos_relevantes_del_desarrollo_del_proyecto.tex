\capitulo{5}{Aspectos relevantes del desarrollo del proyecto}
En este proyecto se presenta una aplicación de escritorio que permite generar cuestionarios sobre análisis ascendente y descendente. Además dicha aplicación permite su posterior publicación en Moodle y en formato \LaTeX{} para su futuro uso en pruebas escritas.


\section{Conocimientos necesarios}
Para realizar esta aplicación ha sido necesario repasar y utilizar algunos de los conocimientos adquiridos durante estos últimos años en el grado.


\begin{itemize}
\item Estructuras de datos.

Uno de los problemas más importantes durante el desarrollo fue la búsqueda de una estructura de datos adecuada para usar en las plantillas. Los conocimientos en estructuras de datos han servido para obtener un software más eficiente y sencillo.
\item Ingeniería de software.

Ha sido utilizada para dar un enfoque sistemático, disciplinado y cuantificable a esta aplicación.

\item Gestión de proyectos.

Se ha seguido una metodología \textit{Scrumm}, que se estudio en esta asignatura. Ha sido necesario repasar los apuntes para seguirla de forma correcta.

\item Procesadores de lenguajes.

Ha sido necesario repasar toda la teoría de dicha asignatura para comprobar que los resultados obtenidos de la aplicación eran correctos.

\end{itemize}  
\section{Conocimientos adquiridos}
Se han adquirido muchos conocimientos durante el desarrollo de \textit{PLGRAM}.
\begin{itemize}
\item Uso de plantillas.

Para generar los documentos se ha utilizado un motor de  plantillas llamado \textit{Mustache}.
\item Archivos .XML.

Los documentos generados con la aplicación \textit{PLGRAM} pueden ser documentos .XML, por lo que se han adquirido conocimientos sobre su diseño y funcionamiento. 
\item Archivos .TEX.

Los documentos generados con la aplicación \textit{PLGRAM} también pueden ser documentos .TEX, por lo que se han adquirido conocimientos sobre su diseño y funcionamiento. 


\item Uso de \TeX{}MAKER

Realizar la documentación del trabajo de fin de grado con \LaTeX{} ha sido un reto que aportará valor para el futuro laboral.


\item Uso de \textit{Pluggins} en Eclipse

La utilización de diferentes pluggins en Eclipse ha facilitado la realización de la aplicación y que el resultado sea una aplicación de calidad.
\end{itemize}

\section{Objetivos realizados}
Durante el desarrollo del proyecto se han perseguido una serie de objetivos que, al final del mismo, han sido cumplidos.

\begin{itemize}
\item Utilidad práctica

Uno de los objetivos del proyecto era la creación de una aplicación la cual tuviera una gran utilizad práctica que permitiese al usuario ahorrarle gran parte de trabajo.
\end{itemize} 