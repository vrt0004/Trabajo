\capitulo{3}{Conceptos teóricos}



\section{Gramática}

El estudio de los lenguajes se puede hacer desde tres puntos de vista:
\begin{itemize}
	\item El de la interpretación: tiene que ver con la semántica de los lenguajes intenta formalizar el significado de las sentencias de un lenguaje.
	
	\item La generación de lenguajes: consiste en encontrar los mecanismos que permitan enumerar las cadenas que pertenecen a un lenguaje. Estos mecanismo son las gramáticas.
	
	
	\item El del reconocimiento del lenguaje: esta muy ligado a la teoría de autómatas y es << ... el estudio de algoritmos o estructuras de máquinas que permiten, dado un lenguaje L y una cadena x, determinar si $ x\in\L$ o  $x\not\in\L$ >> \footnote{ Fernández,1995} 
	
	
\end{itemize}
Una gramática formal es una cuádrupla$(\Sigma,N,P,S)$, donde:
\begin{itemize}
\item $\Sigma$ es un alfabero(conjunto finito y no vacío) simbolos terminales(o tokens/tokenes).
\item $N$ es un alfabeto cuyos elementos se llaman símbolos no terminales.
\item P es un alfabeto de producciones de la forma:


$\lbrace u\rightarrow v:  u=xAy\in(\Sigma\cup N)^{+}\wedge A \in N\wedge v\in(\Sigma\cup N)^{*}\rbrace$
\item S$\in N$ es un símbolo especial llamado axioma o símbolo inicial
\end{itemize}

\
\subsection{Producción}
Una producción o regla de reescritura es un par ordenado de cadenas sobre un alfabeto((x,y):x,y$ \in\Sigma^{\ast}$). Se representa por $x\rightarrow y$. x es la parte izquierda o antecedente de la producción e y es la parte derecha o consecuente.
\subsection{Derivacion directa}
Sea $x\rightarrow y$ una producción y $v,w\in\Sigma^{*}$. Se dice que w deriva directamente de v y se escribe $v\Rightarrow w$ si y sólo si existen $z,u\in\Sigma^{*}$ tales que $v=zxu$, $w=zyu$ y $x\rightarrow y$. 
\subsection{Derivacion (En uno o más pasos)}

w deriva de v y se escribe $v\Rightarrow^{+}w$ cuando existen $u_{0},u_{1},...,u_{n}\in\Sigma^{*}$ tales que:

$v=u_{0}$

$u_{0}\Rightarrow u_{1}$

$u_{1}\Rightarrow u_{2}$

$\vdots$

$u_{n-1}\Rightarrow u_{n}$

$u_{n1}=w$

A la secuencia $u_{0},u_{1},...,u_{n}$ se la llama cadena de dericación de longitud n.










\imagen{thompson-vacio}{Autómata para una expresión vacía}



\section{Listas de items}

Existen tres posibilidades:

\begin{itemize}
	\item primer item.
	\item segundo item.
\end{itemize}

\begin{enumerate}
	\item primer item.
	\item segundo item.
\end{enumerate}

\begin{description}
	\item[Primer item] más información sobre el primer item.
	\item[Segundo item] más información sobre el segundo item.
\end{description}
	


\section{Tablas}

Igualmente se pueden usar los comandos específicos de \LaTeX o bien usar alguno de los comandos de la plantilla.

\tablaSmall{Herramientas y tecnologías utilizadas en cada parte del proyecto}{l c c c c}{herramientasportipodeuso}
{ \multicolumn{1}{l}{Herramientas} & App AngularJS & API REST & BD & Memoria \\}{ 
HTML5 & X & & &\\
CSS3 & X & & &\\
BOOTSTRAP & X & & &\\
JavaScript & X & & &\\
AngularJS & X & & &\\
Bower & X & & &\\
PHP & & X & &\\
Karma + Jasmine & X & & &\\
Slim framework & & X & &\\
Idiorm & & X & &\\
Composer & & X & &\\
JSON & X & X & &\\
PhpStorm & X & X & &\\
MySQL & & & X &\\
PhpMyAdmin & & & X &\\
Git + BitBucket & X & X & X & X\\
Mik\TeX{} & & & & X\\
\TeX{}Maker & & & & X\\
Astah & & & & X\\
Balsamiq Mockups & X & & &\\
VersionOne & X & X & X & X\\
} 
