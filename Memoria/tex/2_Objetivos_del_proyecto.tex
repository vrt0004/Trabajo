\capitulo{2}{Objetivos del proyecto}

\section{Objetivos técnicos}

Los objetivos técnicos desde el punto de vista de desarrollo del programa son:
\begin{enumerate}
	\item Desarrollo de una interfaz gráfica.
	\item Procesado y manipulación de gramáticas. mediante el uso de un procesador de lenguajes.
	\item Cálculo del first y el follow de una gramática.
	\item Cálculo de la tabla de análisis sintáctico predictivo.
	\item Exportación de cuestionarios en formato moodle(XML)
	\item Exportación de cuestionarios en formato LaTeX
	\item Creación de un manual de usuario detallando todas las opciones de la aplicación
\end{enumerate}

\section{Objetivos académicos}
Tal y como se indica en la guía docente, los objetivos academicos mas destacables son los siguientes:
\begin{enumerate}
	
	\item Desarrollar un trabajo personal donde se 	apliquen los conocimientos teóricos y prácticos adquiridos en la titulación.
	\item Ampliar la capacidad creativa mediante el planteamiento y resolución de un problema real.
	\item Aprendizaje autónomo en nuevos temas vinculados a la Ing. Informática.
	\item Capacidad de exposición en público, defensa y argumentación de las elecciones tomadas.
	\item Comunicar correctamente en otro idioma un resumen coherente del trabajo realizado.

\end{enumerate}

\section{Objetivos personales}


El uso de VersionOne\footnote{\url{http://www.VersionOne.com/}} me ha permitido organizarme y poder seguir un control sobre las tareas que debía realizar,he realizado sesiones(sprint) de una semana que han coincidido con las reuniones con los tutores.

El uso de GitHub\footnote{\url{http://www.GitHub.com/}} me ha permitido llevar el control de versiones desde el inicio del trabajo.

