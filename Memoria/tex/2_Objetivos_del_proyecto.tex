\capitulo{2}{Objetivos del proyecto}

\section{Objetivos técnicos}

Los principales objetivos técnicos perseguidos mediante el desarrollo de la aplicación han sido:
\begin{enumerate}
	\item Construcción de una GUI (Interfaz Gráfica de Usuario) que permita visualizar de forma rápida la aplicación desarrollada.
	\item Construcción de una CLI (Command Line Interface) que permita trabajar de forma rápida y directa con la aplicación desarrollada.
	\item Procesado y manipulación de gramáticas mediante procesadores del lenguaje.
	\item Obtención del $FIRST$ y del $FOLLOW$ de los no terminales de una gramática.
	\item Obtención de la Tabla de Análisis Sintáctico Predictivo ($TASP$) de una gramática.
	\item Obtención de los conjuntos de items $LR(0)$ y  $LR(1)$.
	\item Obtención de las tablas de análisis sintáctico  \textit{ACCIÓN e IR A} para análisis $SLR(1)$, $LR(1)$ y $LALR(1)$.
	\item Exportación de cuestionarios en formato \textit{XML} compatible con la plataforma Moodle.
	
	\item Exportación de cuestionarios en formato \LaTeX{}.
	
	\item Creación de un manual de usuario detallando todas las opciones de la aplicación.
\end{enumerate}

\section{Objetivos académicos}
Tal y como se indica en la guía docente del Trabajo Fin de Grado, algunos de los objetivos académicos que se pretenden alcanzar con el mismo son los siguientes:
\begin{enumerate}
	
	\item Desarrollo de un trabajo personal donde se 	apliquen los conocimientos teóricos y prácticos adquiridos en la titulación.
	\item Desarrollo de  la capacidad creativa mediante el planteamiento y resolución de un problema real.
	
	\item Uso y aprendizaje de nuevas herramientas y aplicaciones vinculados a la Ingeniería Informática.

	\item Desarrollo de la capacidad de exposición en público y  defensa y argumentación del trabajo realizado.
	\item Comunicar correctamente en otro idioma un resumen coherente del trabajo realizado.

\end{enumerate}

\section{Objetivos personales}


El uso de VersionOne\footnote{\url{http://www.VersionOne.com/}} me ha permitido organizar y mantener un control sobre las tareas que debía realizar. He realizado sesiones (\textit{sprint}) de una semana que han coincidido con las reuniones con los tutores.

El uso de GitHub\footnote{\url{http://www.GitHub.com/}} me ha permitido llevar el control de versiones desde el inicio del trabajo.

He podido profundizar en el funcionamiento de ciertos aspectos de la máquina virtual de Java: sus librerías nativas, sus métodos de carga dinámica de librerías, paquetes y clases y sus efectos específicos sobre procesos ya en ejecución.

He aprendido a aprovechar las ventajas del lenguaje de programación Java\footnote{\url{http://www.java.com.es/}}: mantener su capacidad multiplataforma evitando realizar llamadas al sistema operativo, sacar partido de sus bibliotecas nativas para la carga de objetos externos y lectura de objetos internos compilados y también explorar las posibilidades de librerías Java de licencia gratuita creadas por su inmensa comunidad de desarrolladores.

