\capitulo{2}{Objetivos del proyecto}

\section{Objetivos técnicos}

El desarrollo de la aplicación ha tenido los siguientes objetivos técnicos :
\begin{enumerate}
	\item Desarrollo de una interfaz gráfica.
	\item Procesado y manipulación de gramáticas mediante procesadores del lenguaje.
	\item Obtención del $FIRTS$ y el $FOLLOW$ de una gramática.
	\item Obtención de la tabla de la tabla de análisis sintáctico predictivo ($TASP$) de una gramática.
	\item Obtención de los conjuntos de items $LR(0)$ y  $LR(1)$.
	\item Obtención de las tablas de análisis sintáctico  \textit{ACCIÓN-GOTO} para análisis $SLR(1)$, $LR(1)$ y $LALR(1)$
	\item Exportación de cuestionarios en formato \textit{XML} compatible con la plataforma Moodle.
	
	\item Exportación de cuestionarios en formato \LaTeX{}.
	
	\item Creación de un manual de usuario detallando todas las opciones de la aplicación
\end{enumerate}

\section{Objetivos académicos}
Tal y como se indica en la guía docente del Trabajo fin de grado, algunos de los objetivos académicos son los siguientes:
\begin{enumerate}
	
	\item Desarrollo de un trabajo personal donde se 	apliquen los conocimientos teóricos y prácticos adquiridos en la titulación.
	\item Desarrollo de  la capacidad creativa mediante el planteamiento y resolución de un problema real.
	
	\item Uso y aprendizaje de nuevas herramientas y aplicaciones vinculados a la Ingeniería Informática.

	\item Desarrollo de la capacidad de exposición en público y  defensa y argumentación del trabajo realizado.
	\item Comunicar correctamente en otro idioma un resumen coherente del trabajo realizado.

\end{enumerate}

\section{Objetivos personales}


El uso de VersionOne\footnote{\url{http://www.VersionOne.com/}} me ha permitido organizarme y poder seguir un control sobre las tareas que debía realizar, he realizado sesiones(sprint) de una semana que han coincidido con las reuniones con los tutores.

El uso de GitHub\footnote{\url{http://www.GitHub.com/}} me ha permitido llevar el control de versiones desde el inicio del trabajo.

He podido profundizar en el funcionamiento de ciertos aspectos de la máquina virtual de java: sus librerías nativas, sus métodos de carga dinámica de librerías, paquetes y clases y sus efectos específicos sobre procesos ya en ejecución.

He aprendido a aprovechar las ventajas del lenguaje de programación Java: mantener su capacidad multiplataforma evitando realizar llamadas al sistema operativo, sacar partido de sus librerías nativas para la carga de objetos externos y lectura de objetos internos compilados y también explorar las posibilidades de librerías java de licencia gratuita creadas por su inmensa comunidad de desarrolladores.

