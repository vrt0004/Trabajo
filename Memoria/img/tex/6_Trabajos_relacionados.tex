\capitulo{6}{Trabajos relacionados}

Para comenzar a trabajar en el desarrollo de la presente aplicación se buscó información relacionada con la aplicación que se quería llevar a cabo.

Para ello, se realizó una investigación a través de la Web sobre  aplicaciones que tuvieran una función similar a la desarrollada. 

Aunque no existe una gran variedad de programas que realicen tareas similares a $PLGRAM$, entre los dos programas recomendados por el tutor se  ha encontrado información suficiente para poder llevar a cabo la aplicación. Dichos programas son $PLQUIZ$\cite{PLQUIZ} y $BURGRAM$\cite{BURGRAM}.

Dado que ambos programas han sido desarrollados por alumnos de la Universidad de Burgos, como trabajos de final de grado, el tutor pudo facilitarme su estructura y toda la documentación que estimó necesaria para la correcta ejecución de $ PLGRAM $.


\section{PLQUIZ\cite{PLQUIZ}}

PLQUIZ\footnote{Roberto Izquierdo Amo, Julio 2014, Universidad de Burgos} es una herramienta de escritorio que permite generar preguntas de test aleatorias (tipo quiz, cloze, de texto libre...) sobre problemas de algoritmos de análisis léxico. El formato utilizado para generar las preguntas es .XML para importarse a entornos virtuales de aprendizaje (Moodle), y .TEX para  su impresión en papel a través de la obtención de código \LaTeX{}.

Se han seguido sus pasos en el formato de la interfaz gráfica, pues tiene una organización muy clara para el usuario, permitiendo acceder rápidamente a las opciones disponibles.

\imagen{PLQUIZ}{PLQUIZ}


\section{BURGRAM\cite{BURGRAM}}

BURGRAM\footnote{Carlos Gómez Palacios, Enero 2008, Universidad de Burgos} es una herramienta que permite seguir paso a paso el proceso de los distintos modos de análisis sintáctico (tanto ascendente como descendente). La aplicación permite editar gramáticas, visualizar el proceso completo de operación del algoritmo sobre las mismas y la generación de informes con los resultados.

Se han reutilizado varias de las clases de este programa pues realizan el análisis que la aplicación desarrollada necesitaba para obtener los datos que posteriormente se han tratado para obtener los ficheros .XML.
\imagen{BURGRAM}{BURGRAM}


\section{SEFALAS\cite{SEFALAS}}
SEFALAS\footnote{José Francisco Jódar Reyes, 2004, Universidad de Granada} (\textbf{S}oftware para la \textbf{E}nseñanza de las \textbf{F}ases de \textbf{Á}nálisis \textbf{L}éxico y \textbf{A}nálisis \textbf{S}intáctico) es una aplicación desarrollada como herramienta para la enseñanza de ciertas técnicas de análisis léxico y de análisis sintáctico. En lo referente a análisis sintáctico SEFALAS recoge las dos estrategias: análisis descendente y análisis ascendente. La primera de ellas se ilustra mediante el método descendente LL(1). La segunda se ilustra con los métodos de precedencia de operador y precedencia simple y las gramáticas LR en sus diversas modalidades: SLR, LR(1) y LALR. Además, es posible escribir la gramática en formato YACC y Sefalas construye la tabla de análisis dependiendo de la estrategia seleccionada. Es posible realizar el análisis de un texto de entrada y ver la evolución del análisis de forma interactiva a través de la tabla de análisis.
\imagen{SEFALAS}{SEFALAS}