\capitulo{1}{Introducción}

\section{Introducción}

Una de las principales características del actual concepto de docencia, sobre todo universitaria, es el gran auge que ha experimentado la docencia online. Esto se ve reflejado no sólo en el apoyo que supone a la docencia presencial, sino en la demanda de educación a distancia que cada vez tiene un mayor público.

Un claro ejemplo es el campus virtual de la Universidad de Burgos. UBUVirtual es una plataforma virtual basada en WebMoodle, que a su vez es una plataforma de eLearning utilizada por una gran cantidad de usuarios.

Por tanto, dotar de todas las herramientas necesarias a dicha plataforma para ofrecer un soporte útil y dinámico tanto al alumno como al profesorado, es una de las prioridades a tener en cuenta. Surge así la necesidad de realizar pruebas y exámenes a través de la plataforma.

En este proyecto se presenta una aplicación de escritorio que permite generar cuestionarios sobre análisis ascendente y descendente. Además dicha aplicación permite su posterior publicación en Moodle y en formato \LaTeX{} para su futuro uso en pruebas escritas.














