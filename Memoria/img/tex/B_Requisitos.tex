\apendice{Especificación de Requisitos}

\section{Introducción}
Este anexo tiene como objetivo analizar y documentar las necesidades funcionales mínimas que deben ser soportadas por la aplicación desarrollada. 

Otro objetivo básico del presente anexo es la de priorizar cada una de las funcionalidades a implementar. Si se tiene claro cuáles son las tareas críticas, se podrá llevar a cabo un enfoque más preciso en ellas y asignarlas una mayor cantidad de recursos.

El anexo se estructura presentado una lista de objetivos generales que la aplicación debe cumplir. Son los objetivos que el cliente establece mediante un lenguaje neutral y es trabajo del ingeniero su desglose y formalización. A continuación se incluye un catálogo de los requisitos que deberá cumplir la aplicación.

\section{Objetivos generales}
El desarrollo de la aplicación cumple los siguientes objetivos:
\begin{enumerate}
	\item Construcción de una \textit{GUI} (Interfaz Gráfica de Usuario) que permita visualizar de forma rápida la aplicación desarrollada.
	\item Construcción de una \textit{CLI} (Command Line Interface) que permita trabajar de forma rápida y directa con la aplicación desarrollada.
	\item Procesado y manipulación de gramáticas mediante procesadores del lenguaje.
	\item Obtención del $FIRST$ y del $FOLLOW$ de una gramática.
	\item Obtención de la Tabla de Análisis Sintáctico Predictivo ($TASP$) de una gramática.
	\item Obtención de los conjuntos de items $LR(0)$ y  $LR(1)$.
	\item Obtención de las tablas de análisis sintáctico  \textit{ACCIÓN e IR A} para análisis $SLR(1)$, $LR(1)$ y $LALR(1)$.
	\item Exportación de cuestionarios en formato .XML compatible con la plataforma Moodle.
	
	\item Exportación de cuestionarios en formato \LaTeX{}.
	
	\item Creación de un manual de usuario detallando todas las opciones de la aplicación.
\end{enumerate}

\section{Catálogo de requisitos}
El objetivo de este apartado es definir de forma clara, precisa, completa y verificable todas las funcionalidades y restricciones del sistema a construir.

\subsection{Funciones del producto}
Una de las características más importantes de la aplicación que se ha  desarrollado es su facilidad de uso. 
Dado que su empleo está orientado a convertirse en una herramienta docente, debe ser fácil de usar y posibilitar la creación de cuestionarios de manera rápida, sencilla y comprensible.

Como se explica en el \textit{Anexo E: Manual de usuario}, los archivos obtenidos de la aplicación tendrán extensiones .TEX y .XML, tratando de cubrir un mayor abanico de posibilidades de uso de la aplicación y dando una mayor cobertura a la hora de obtener los cuestionarios.

En el caso de los archivos .XML es necesario el uso de la plataforma \textit{Moodle}, que al ser internacional, se puede utilizar en varios idiomas, dándole así una mayor difusión a la aplicación.

\subsection{Requisitos de usuario}
El usuario debe ser capaz de utilizar la aplicación mediante la Interfaz Gráfica o mediante la Consola del sistema. 

La ayuda será un punto a tener en cuenta en el caso de utilizar la aplicación mediante la consola del sistema.
 
En todo momento el usuario podrá consultar el \textit{Manual del Usuario} para
orientarse y resolver dudas que le puedan surgir durante el manejo de la misma.

\subsection{Requisitos del sistema}
Los requisitos mínimos del hardware del sistema serán los mismos que para ejecutar java\cite{requisitos}.


Los requisitos software necesarios son:
\begin{itemize}
\item Sistema operativo \textit{Windows}, \textit{Linux} o \textit{Macintosh}.
\item Máquina virtual de java ($JDK 8$).
\item Consola de sistema operativo capaz de interpretar Unicode, solo para la interfaz en línea de comandos. Si no soportara Unicode, las tildes u otros símbolos podrían ser mal representados.
\end{itemize}

\section{Especificación de requisitos}
En este apartado se realizará un análisis de los requisitos de diseño de la aplicación. Para ello, se detallarán los requisitos funcionales y no funcionales de la aplicación.

\subsection{Requisitos funcionales}
La aplicación debe ser capaz de cumplir los siguientes requisitos:

\begin{itemize}
\item \textit{RF1: Cargar gramática y análisis}. Permite al usuario seleccionar una gramática disponible en la carpeta \textit{gramaticas} y un tipo de análisis.
\item \textit{RF2: Obtener análisis}. Se obtienen los datos del análisis solicitado para la gramática seleccionada.
\item \textit{RF3: Crear cadenas}. Se obtienen las cadenas de los datos del análisis.
\item \textit{RF4: Obtener fichero} .TEX. Se utiliza la plantilla para obtener el fichero .TEX.
\item \textit{RF5: Obtener fichero} .XML. Se utiliza la plantilla para obtener el fichero .XML.
\end{itemize}



\begin{tabular}{| p{3.2cm}| p{8.2cm} |}
\hline  \multicolumn{2}{|c|} {RF1 Cargar gramática y análisis}\\ 
\hline
\hline    Descripción        & Permite al usuario seleccionar una gramática y un tipo de análisis. \\ 
\hline    Precondiciones     &   Que exista la gramática.       \\ 
\hline    Secuencia normal   &   Introducir gramática y análisis.       \\ 
\hline    Postcondiciones    &          \\ 
\hline    Excepciones        &   Si no existe la gramática o tiene algún error.       \\ 
\hline    Rendimiento        &          \\ 
\hline    Frecuencia         &     Alto.     \\ 
\hline    Importancia        &       Alto.   \\ 
\hline    Urgencia    		 &    Alto.      \\ 
\hline
\end{tabular}

\vspace{0,5cm}

\begin{tabular}{| p{3.2cm}| p{8.2cm} |}
\hline  \multicolumn{2}{|c|} {RF2 Obtener análisis}\\ 
\hline
\hline    Descripción        &   Se obtienen los datos del análisis para la gramática seleccionada.\\ 
\hline    Precondiciones     &   Tipo de análisis valido.       \\ 
\hline    Secuencia normal   &   Introducir análisis.       \\ 
\hline    Postcondiciones    &          \\ 
\hline    Excepciones        &   Si no existe el análisis muestra un error.        \\ 
\hline    Rendimiento        &          \\ 
\hline    Frecuencia         &     Alto.     \\ 
\hline    Importancia        &       Alto.   \\ 
\hline    Urgencia    		 &    Alto.      \\ 
\hline
\end{tabular}

\vspace{0,5cm}

\begin{tabular}{| p{3.2cm}| p{8.2cm} |}
\hline  \multicolumn{2}{|c|} {RF3 crear cadenas}\\ 
\hline
\hline    Descripción        & Se obtienen las cadenas del análisis para la gramática seleccionada.\\ 
\hline    Precondiciones     &          \\ 
\hline    Secuencia normal   &         \\ 
\hline    Postcondiciones    &          \\ 
\hline    Excepciones        &   \\ 
\hline    Rendimiento        &          \\ 
\hline    Frecuencia         &    Alto.     \\ 
\hline    Importancia        &    Alto.   \\ 
\hline    Urgencia    		 &    Alto.      \\ 
\hline
\end{tabular}

\vspace{0,5cm}

\begin{tabular}{| p{3.2cm}| p{8.2cm} |}
\hline  \multicolumn{2}{|c|} {RF4 Obtener fichero .TEX}\\ 
\hline
\hline    Descripción        & Se utiliza la plantilla para obtener el fichero .TEX. \\ 
\hline    Precondiciones     &    Que exista la plantilla PlantillaTEX.     \\ 
\hline    Secuencia normal   &         \\ 
\hline    Postcondiciones    &          \\ 
\hline    Excepciones        &    \\ 
\hline    Rendimiento        &          \\ 
\hline    Frecuencia         &    Medio.     \\ 
\hline    Importancia        &    Alto.   \\ 
\hline    Urgencia    		 &    Alto.      \\ 
\hline
\end{tabular}



\vspace{0,5cm}

\begin{tabular}{| p{3.2cm}| p{8.2cm} |}
\hline  \multicolumn{2}{|c|} {RF5 Obtener fichero .XML}\\ 
\hline
\hline    Descripción        & Se utiliza la plantilla para obtener el fichero .XML. \\ 
\hline    Precondiciones     &    Que exista la plantilla Plantillamoodle.       \\ 
\hline    Secuencia normal   &          \\ 
\hline    Postcondiciones    &          \\ 
\hline    Excepciones        &          \\ 
\hline    Rendimiento        &          \\ 
\hline    Frecuencia         &     Medio.     \\ 
\hline    Importancia        &     Alto   \\ 
\hline    Urgencia    		 &     Alto      \\ 
\hline
\end{tabular}
\imagen{diagramaprincipal}{Diagrama principal de casos de uso.}

\subsection{Requisitos no funcionales}
A continuación se describen todos aquellos requisitos no funcionales importantes para considerar en el diseño:


\begin{itemize}
\item \textit{RNF1: Eficiencia}. Se debe buscar la eficiencia de la aplicación para que pueda ser usada con gramáticas complejas.
\item \textit{RNF2: Extensibilidad}. Se pueden definir nuevas capacidades en la aplicación mediante su extensión a otro tipo de ficheros.
\item \textit{RNF3: Práctica}. Al ser una aplicación docente, un requisito es que sea útil y práctica para enseñar/evaluar
\end{itemize}