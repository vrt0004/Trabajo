\documentclass[a4paper,11pt,twoside,openright]{memoir}

% Castellano
\usepackage[spanish]{babel}
\selectlanguage{spanish}
\usepackage[utf8]{inputenc}
\usepackage{placeins}

\RequirePackage{booktabs}
\RequirePackage[table]{xcolor}
\RequirePackage{xtab}
\RequirePackage{multirow}

% Links
\usepackage[colorlinks]{hyperref}
\hypersetup{
	allcolors = {red}
}

% Ecuaciones
\usepackage{amsmath}

% Rutas de fichero / paquete
\newcommand{\ruta}[1]{{\sffamily #1}}

% Párrafos
\nonzeroparskip


% Imagenes
\usepackage{graphicx}
\newcommand{\imagen}[2]{
	\begin{figure}[!h]
		\centering
		\includegraphics[width=0.9\textwidth]{#1}
		\caption{#2}\label{fig:#1}
	\end{figure}
	\FloatBarrier
}

\newcommand{\imagenflotante}[2]{
	\begin{figure}%[!h]
		\centering
		\includegraphics[width=0.9\textwidth]{#1}
		\caption{#2}\label{fig:#1}
	\end{figure}
}



% El comando \figura nos permite insertar figuras comodamente, y utilizando
% siempre el mismo formato. Los parametros son:
% 1 -> Porcentaje del ancho de página que ocupará la figura (de 0 a 1)
% 2 --> Fichero de la imagen
% 3 --> Texto a pie de imagen
% 4 --> Etiqueta (label) para referencias
% 5 --> Opciones que queramos pasarle al \includegraphics
% 6 --> Opciones de posicionamiento a pasarle a \begin{figure}
\newcommand{\figuraConPosicion}[6]{%
  \setlength{\anchoFloat}{#1\textwidth}%
  \addtolength{\anchoFloat}{-4\fboxsep}%
  \setlength{\anchoFigura}{\anchoFloat}%
  \begin{figure}[#6]
    \begin{center}%
      \Ovalbox{%
        \begin{minipage}{\anchoFloat}%
          \begin{center}%
            \includegraphics[width=\anchoFigura,#5]{#2}%
            \caption{#3}%
            \label{#4}%
          \end{center}%
        \end{minipage}
      }%
    \end{center}%
  \end{figure}%
}

%
% Comando para incluir imágenes en formato apaisado (sin marco).
\newcommand{\figuraApaisadaSinMarco}[5]{%
  \begin{figure}%
    \begin{center}%
    \includegraphics[angle=90,height=#1\textheight,#5]{#2}%
    \caption{#3}%
    \label{#4}%
    \end{center}%
  \end{figure}%
}
% Para las tablas
\newcommand{\otoprule}{\midrule [\heavyrulewidth]}
%
% Nuevo comando para tablas pequeñas (menos de una página).
\newcommand{\tablaSmall}[5]{%
 \begin{table}
  \begin{center}
   \rowcolors {2}{gray!35}{}
   \begin{tabular}{#2}
    \toprule
    #4
    \otoprule
    #5
    \bottomrule
   \end{tabular}
   \caption{#1}
   \label{tabla:#3}
  \end{center}
 \end{table}
}

%
% Nuevo comando para tablas pequeñas (menos de una página).
\newcommand{\tablaSmallSinColores}[5]{%
 \begin{table}[H]
  \begin{center}
   \begin{tabular}{#2}
    \toprule
    #4
    \otoprule
    #5
    \bottomrule
   \end{tabular}
   \caption{#1}
   \label{tabla:#3}
  \end{center}
 \end{table}
}

\newcommand{\tablaApaisadaSmall}[5]{%
\begin{landscape}
  \begin{table}
   \begin{center}
    \rowcolors {2}{gray!35}{}
    \begin{tabular}{#2}
     \toprule
     #4
     \otoprule
     #5
     \bottomrule
    \end{tabular}
    \caption{#1}
    \label{tabla:#3}
   \end{center}
  \end{table}
\end{landscape}
}

%
% Nuevo comando para tablas grandes con cabecera y filas alternas coloreadas en gris.
\newcommand{\tabla}[6]{%
  \begin{center}
    \tablefirsthead{
      \toprule
      #5
      \otoprule
    }
    \tablehead{
      \multicolumn{#3}{l}{\small\sl continúa desde la página anterior}\\
      \toprule
      #5
      \otoprule
    }
    \tabletail{
      \hline
      \multicolumn{#3}{r}{\small\sl continúa en la página siguiente}\\
    }
    \tablelasttail{
      \hline
    }
    \bottomcaption{#1}
    \rowcolors {2}{gray!35}{}
    \begin{xtabular}{#2}
      #6
      \bottomrule
    \end{xtabular}
    \label{tabla:#4}
  \end{center}
}

%
% Nuevo comando para tablas grandes con cabecera.
\newcommand{\tablaSinColores}[6]{%
  \begin{center}
    \tablefirsthead{
      \toprule
      #5
      \otoprule
    }
    \tablehead{
      \multicolumn{#3}{l}{\small\sl continúa desde la página anterior}\\
      \toprule
      #5
      \otoprule
    }
    \tabletail{
      \hline
      \multicolumn{#3}{r}{\small\sl continúa en la página siguiente}\\
    }
    \tablelasttail{
      \hline
    }
    \bottomcaption{#1}
    \begin{xtabular}{#2}
      #6
      \bottomrule
    \end{xtabular}
    \label{tabla:#4}
  \end{center}
}

%
% Nuevo comando para tablas grandes sin cabecera.
\newcommand{\tablaSinCabecera}[5]{%
  \begin{center}
    \tablefirsthead{
      \toprule
    }
    \tablehead{
      \multicolumn{#3}{l}{\small\sl continúa desde la página anterior}\\
      \hline
    }
    \tabletail{
      \hline
      \multicolumn{#3}{r}{\small\sl continúa en la página siguiente}\\
    }
    \tablelasttail{
      \hline
    }
    \bottomcaption{#1}
  \begin{xtabular}{#2}
    #5
   \bottomrule
  \end{xtabular}
  \label{tabla:#4}
  \end{center}
}



\definecolor{cgoLight}{HTML}{EEEEEE}
\definecolor{cgoExtralight}{HTML}{FFFFFF}

%
% Nuevo comando para tablas grandes sin cabecera.
\newcommand{\tablaSinCabeceraConBandas}[5]{%
  \begin{center}
    \tablefirsthead{
      \toprule
    }
    \tablehead{
      \multicolumn{#3}{l}{\small\sl continúa desde la página anterior}\\
      \hline
    }
    \tabletail{
      \hline
      \multicolumn{#3}{r}{\small\sl continúa en la página siguiente}\\
    }
    \tablelasttail{
      \hline
    }
    \bottomcaption{#1}
    \rowcolors[]{1}{cgoExtralight}{cgoLight}

  \begin{xtabular}{#2}
    #5
   \bottomrule
  \end{xtabular}
  \label{tabla:#4}
  \end{center}
}


















\graphicspath{ {./img/} }

% Capítulos
\chapterstyle{bianchi}
\newcommand{\capitulo}[2]{
	\setcounter{chapter}{#1}
	\setcounter{section}{0}
	\chapter*{#2}
	\addcontentsline{toc}{chapter}{#2}
	\markboth{#2}{#2}
}

% Apéndices
\renewcommand{\appendixname}{Apéndice}
\renewcommand*\cftappendixname{\appendixname}

\newcommand{\apendice}[1]{
	%\renewcommand{\thechapter}{A}
	\chapter{#1}
}

\renewcommand*\cftappendixname{\appendixname\ }

% Formato de portada
\makeatletter
\usepackage{xcolor}
\newcommand{\tutor}[1]{\def\@tutor{#1}}
\newcommand{\course}[1]{\def\@course{#1}}
\definecolor{cpardoBox}{HTML}{E6E6FF}
\def\maketitle{
  \null
  \thispagestyle{empty}
  % Cabecera ----------------
\noindent\includegraphics[width=\textwidth]{cabecera}\vspace{1cm}%
  \vfill
  % Título proyecto y escudo informática ----------------
  \colorbox{cpardoBox}{%
    \begin{minipage}{.8\textwidth}
      \vspace{.5cm}\Large
      \begin{center}
      \textbf{TFG del Grado en Ingeniería Informática}\vspace{.6cm}\\
      \textbf{\LARGE\@title{}}
      \end{center}
      \vspace{.2cm}
    \end{minipage}

  }%
  \hfill\begin{minipage}{.20\textwidth}
    \includegraphics[width=\textwidth]{escudoInfor}
  \end{minipage}
  \vfill
  % Datos de alumno, curso y tutores ------------------
  \begin{center}%
  {%
    \noindent\LARGE
    Presentado por \@author{}\\ 
    en Universidad de Burgos --- \@date{}\\
    Tutores: \@tutor{}\\
  }%
  \end{center}%
  \null
  \newpage
  \thispagestyle{empty}
  \newpage
  }
\makeatother


% Datos de portada
\title{PLGRAM}
\author{Víctor Renuncio Tobar}
\tutor{Dr. César Ignacio García Osorio y D. Álvar Arnaiz González}
\date{\today}

\begin{document}

\maketitle



\null\cleardoublepage


%%%%%%%%%%%%%%%%%%%%%%%%%%%%%%%%%%%%%%%%%%%%%%%%%%%%%%%%%%%%%%%%%%%%%%%%%%%%%%%%%%%%%%%%
\thispagestyle{empty}


\noindent\includegraphics[width=\textwidth]{cabecera}\vspace{1cm}

\noindent D. César Ignacio García Osorio, profesor del departamento de Ingeniería Civil, área de Lenguajes y Sistemas Informáticos y D. Álvar Arnaiz González, profesor del departamento de Ingeniería Civil, área de Lenguajes y Sistemas Informáticos

\noindent Exponen:

\noindent Que el alumno D. Víctor Renuncio Tobar, con DNI 71.264.303-E, ha realizado el Trabajo final de Grado en Ingeniería Informática titulado PLGRAM. 

\noindent Y que dicho trabajo ha sido realizado por el alumno bajo la dirección de los que suscriben, en virtud de lo cual se autoriza su presentación y defensa.

\begin{center} %\large
En Burgos, {\large \today}
\end{center}

\vfill\vfill\vfill

% Author and supervisor
\begin{minipage}{0.45\textwidth}
\begin{flushleft} %\large
Vº. Bº. del Tutor:\\[2cm]
D. Cesar Ignacio García Osorio
\end{flushleft}
\end{minipage}
\hfill
\begin{minipage}{0.45\textwidth}
\begin{flushleft} %\large
Vº. Bº. del co-tutor:\\[2cm]
D. Álvar Arnaiz González
\end{flushleft}
\end{minipage}
\hfill

\vfill

% para casos con solo un tutor comentar lo anterior
% y descomentar lo siguiente
%Vº. Bº. del Tutor:\\[2cm]
%D. nombre tutor


\newpage\thispagestyle{empty}
\null




\frontmatter

% Abstract en castellano
\renewcommand*\abstractname{Resumen}
\begin{abstract}
Desarrollo de una aplicación que a partir de una gramática obtiene el análisis sintáctico descendente \textit{LL(1)} y los análisis sintácticos ascendente \textit{SLR(1)}, \textit{LR(1)} y \textit{LALR(1)}. Se obtienen los conjuntos \textit{FIRST} y \textit{FOLLOW} así como la tabla de análisis sintáctico predictivo y las tablas de \textit{ACCIÓN y de IR A}. Asimismo se puede obtener para los distintos análisis la traza de una cadena.

Con la aplicación se puede generar un archivo .XML para importar a la plataforma \textit{Moodle} con la finalidad de generar de forma automática cuestionarios de análisis sintáctico. También se puede generar un fichero en formato \LaTeX{} que se puede utilizar para realizar pruebas escritas. Se puede obtener un documento en el que el cuestionario esté completo, con las respuestas correctas y otro documento en el que el cuestionario está vacío, pensado para ser respondido por un tercero.
\end{abstract}
\renewcommand*\abstractname{Descriptores}
\begin{abstract}
Generación de cuestionarios, análisis sintáctico ascendente, análisis sintáctico descendente, \textit{LL}(1), \textit{SLR}(1), \textit{LALR}(1), \textit{LR}(1).
\end{abstract}

\clearpage

% Abstract en inglés
\renewcommand*\abstractname{Abstract}
\begin{abstract}

Development of an application that obtains top-down parsing \textit{LL(1)} and botton-up parsing \textit{SLR(1)}, LR(1) and LALR(1) from a grammar. Obteins the \textit{FIRST} and \textit{FOLLOW} sets and the predictive parsing table and tables \textit{ACTION} and \textit{GOTO} .Also it can be obtained for each diferent analyzes the trace of a string .

With the application you can generate a .XML file to import to the Moodle platform in order to generate automatically parse questionnaires. You can also generate a \LaTeX{} file format that can be used for written tests. You can get a document that the questionnaire is completed, with the correct answers and other documents in which the questionnaire is empty, designed to be solved by a third party.

\end{abstract}

\renewcommand*\abstractname{Keywords}
\begin{abstract}
Questionnaire generation, bottom-up parsing, top-down parsing, \textit{LL}(1), \textit{SLR}(1), \textit{LALR}(1), \textit{LR}(1).
\end{abstract}

\clearpage

% Indices
\tableofcontents

\clearpage

\listoffigures

\clearpage

%\listoftables

%\clearpage

\mainmatter
\capitulo{1}{Introducción}

\section{Introducción}
En los últimos años ha tenido gran auge la docencia online, tanto como apoyo a la docencia presencial como en docencia a distancia.

El campus virtual de la Universidad de Burgos,  UBUVirtual,  está basado en la  plataforma Web  Moodle que es  una plataforma de eLearning utilizada por gran cantidad de usuarios.

En este proyecto presento una aplicación de escritorio que permite generar cuestionarios sobre análisis ascendente y descendente,  para su posterior publicación en Moodle 











\capitulo{2}{Objetivos del proyecto}

\section{Objetivos técnicos}

Los principales objetivos técnicos perseguidos mediante el desarrollo de la aplicación han sido:
\begin{enumerate}
	\item Construcción de una GUI (Interfaz Gráfica de Usuario) que permita visualizar de forma rápida la aplicación desarrollada.
	\item Construcción de una CLI (Command Line Interface) que permita trabajar de forma rápida y directa con la aplicación desarrollada.
	\item Procesado y manipulación de gramáticas mediante procesadores del lenguaje.
	\item Obtención del $FIRST$ y del $FOLLOW$ de una gramática.
	\item Obtención de la Tabla de Análisis Sintáctico Predictivo ($TASP$) de una gramática.
	\item Obtención de los conjuntos de items $LR(0)$ y  $LR(1)$.
	\item Obtención de las tablas de análisis sintáctico  \textit{ACCIÓN e IR A} para análisis $SLR(1)$, $LR(1)$ y $LALR(1)$.
	\item Exportación de cuestionarios en formato \textit{XML} compatible con la plataforma Moodle.
	
	\item Exportación de cuestionarios en formato \LaTeX{}.
	
	\item Creación de un manual de usuario detallando todas las opciones de la aplicación.
\end{enumerate}

\section{Objetivos académicos}
Tal y como se indica en la guía docente del Trabajo Fin de Grado, algunos de los objetivos académicos que se pretenden alcanzar con el mismo son los siguientes:
\begin{enumerate}
	
	\item Desarrollo de un trabajo personal donde se 	apliquen los conocimientos teóricos y prácticos adquiridos en la titulación.
	\item Desarrollo de  la capacidad creativa mediante el planteamiento y resolución de un problema real.
	
	\item Uso y aprendizaje de nuevas herramientas y aplicaciones vinculados a la Ingeniería Informática.

	\item Desarrollo de la capacidad de exposición en público y  defensa y argumentación del trabajo realizado.
	\item Comunicar correctamente en otro idioma un resumen coherente del trabajo realizado.

\end{enumerate}

\section{Objetivos personales}


El uso de VersionOne\footnote{\url{http://www.VersionOne.com/}} me ha permitido organizar y mantener un control sobre las tareas que debía realizar. He realizado sesiones (\textit{sprint}) de una semana que han coincidido con las reuniones con los tutores.

El uso de GitHub\footnote{\url{http://www.GitHub.com/}} me ha permitido llevar el control de versiones desde el inicio del trabajo.

He podido profundizar en el funcionamiento de ciertos aspectos de la máquina virtual de Java: sus librerías nativas, sus métodos de carga dinámica de librerías, paquetes y clases y sus efectos específicos sobre procesos ya en ejecución.

He aprendido a aprovechar las ventajas del lenguaje de programación Java\footnote{\url{http://www.java.com.es/}}: mantener su capacidad multiplataforma evitando realizar llamadas al sistema operativo, sacar partido de sus librerías nativas para la carga de objetos externos y lectura de objetos internos compilados y también explorar las posibilidades de librerías Java de licencia gratuita creadas por su inmensa comunidad de desarrolladores.


\capitulo{3}{Conceptos teóricos}
\section{Gramática}

El estudio de los lenguajes se puede hacer desde tres puntos de vista:
\begin{itemize}
	\item El de la interpretación: tiene que ver con la semántica de los lenguajes intenta formalizar el significado de las sentencias de un lenguaje.
	
	\item La generación de lenguajes: consiste en encontrar los mecanismos que permitan enumerar las cadenas que pertenecen a un lenguaje. Estos mecanismo son las gramáticas.
	
	
	\item El del reconocimiento del lenguaje: esta muy ligado a la teoría de autómatas y es << ... el estudio de algoritmos o estructuras de máquinas que permiten, dado un lenguaje L y una cadena x, determinar si $ x\in L$ o  $x\not\in L$ >>. \footnote{ Fernández,1995} 
	
	
\end{itemize}

Una gramática formal es una cuádrupla$(\Sigma,N,P,S)$, donde:

\begin{itemize}
\item $\Sigma$ es un alfabeto(conjunto finito y no vacío) símbolos terminales(o tokens/tokenes).
\item $N$ es un alfabeto cuyos elementos se llaman símbolos no terminales.
\item P es un alfabeto de producciones de la forma:


$\lbrace u\rightarrow v:  u=xAy\in(\Sigma\cup N)^{+}\wedge A \in N\wedge v\in(\Sigma\cup N)^{*}\rbrace$
\item S$\in N$ es un símbolo especial llamado axioma o símbolo inicial
\end{itemize}

\
\subsection{Producción}
Una producción o regla de re-escritura es un par ordenado de cadenas sobre un alfabeto((x,y):x,y$ \in\Sigma^{\ast}$). Se representa por $x\rightarrow y$. x es la parte izquierda o antecedente de la producción e y es la parte derecha o consecuente.
\subsection{Derivación directa}
Sea $x\rightarrow y$ una producción y $v,w\in\Sigma^{*}$. Se dice que w deriva directamente de v y se escribe $v\Rightarrow w$ si y sólo si existen $z,u\in\Sigma^{*}$ tales que $v=zxu$, $w=zyu$ y $x\rightarrow y$. 
\subsection{Derivación (En uno o más pasos)}

w deriva de v y se escribe $v\Rightarrow^{+}w$ cuando existen $u_{0},u_{1},...,u_{n}\in\Sigma^{*}$ tales que:
$v=u_{0}$
$u_{0}\Rightarrow u_{1}$
$u_{1}\Rightarrow u_{2}$
$\vdots$
$u_{n-1}\Rightarrow u_{n}$
$u_{n1}=w$

A la secuencia $u_{0},u_{1},...,u_{n}$ se la llama cadena de derivación de longitud n.


En el contexto de las gramáticas el alfabeto ca a estar formado por la unión de los alfabetos de terminales y no terminales.Ademas el antecedente de una producción nunca va a poder ser epsilon y ha de contener al menos un no terminal. 

\subsection{Derivación más a la izquierda}
Cuando las producciones utilizadas en la derivación se aplican siempre a los símbolos más a la izquierda.

\subsection{Derivación más a la derecha}
Cuando las producciones se aplican a los símbolos más a la derecha.
\subsection{Forma sentencial}
Dada una gramática $(\Sigma,N,P,S)$. una cadena $\alpha\in(\Sigma\cupN)*$ es una forma sentencial de esa gramática si existe una derivación que produce $\alpha$ a partir del axioma S, es decir si $S\Rightarrow*a$.
\subsection{Frase o sentencia}
Es una forma sentencial $\alpha$ que solo contiene símbolos terminales($\alpha\in\Sigma* $).
\subsection{Lenguaje generado por una gramática}
El lenguaje generado por una gramática G se representa por L(G) y se define como el conjunto de todas las sentencias de la gramática G. 
$L(G)={x\in\Sigma*:S\Rightarrow ^{+} x}$
\subsection{Gramáticas equivalentes}
Se dice que dos gramáticas son equivalentes si generan el mismo lenguaje $L(G{\small 1})=L(G{\small 2})$.
Se representa por $G{\small 1}\equivG{\small 2}$
\subsection{Gramáticas recursiva en un cierto símbolo no terminal U}
Cuando existe una forma sentencial de U que contiene a U.
$U\Rightarrow^{+}xUy con x,y\in(\Sigma\cupN)^{*}$
La gramática será recursiva si es recursiva para algún no terminal.
\begin{itemize}
	\item Si $x=\varepsilon$ se dice que es una gramática recursiva a izquierdas.
	\item Si $y=\varepsilon$ se dice que es una gramática recursiva a derechas.
\end{itemize}



\section{Clasificación de las gramáticas}
\subsection{Gramáticas de tipo 0}

Las gramáticas de tipo 0 o gramáticas de Chomsky, con reglas de producción de la forma:
$u\rightarrowv con u=xAy \in (\Sigma\cupN)^{+}\wedgeA\inN\wedgex,y,v \in(\Sigma\cupN)^{*}$
El conjunto de lenguajes de tipo 0 coincide con el de todos los lenguajes gramaticales posibles.
Puede demostrarse que todo lenguaje generado por una gramatica de Chomsky puede generarse tambien por unas gramáticas más restrictivas llamadas gramáticas con estructura de frase, cuyas reglas de producción son de la forma:
$xAy \rightarrowxvy con x,y,v\in(\Sigma\cupN)^{*}\wedgeA\inN$
\subsection{Gramáticas de tipo 1}

Tambien llamadas sensibles al contexto, con reglas de producción de la forma:
$xAy \rightarrowxvy con x,y\in(\Sigma\cupN)^{*}\wedgeA\inN\wedgev\in(\Sigma\cupN)^{+}$
En los lenguajes generados por estas gramáticas el significado de las "palabras" depende de su posición en la frase.
A la x e y ses a lo que se le llama contexto( es decir, A sólo puede transformarse en v si va precedido de x y al mismo tiempo seguido de y).
No tiene reglas compresoras, aunque se tolera la regla $S\rightarrow\varepsilon\varepsilon$.
Son las gramáticas de mayor categoría que se suelen utilizar (la mayor parte de los lenguajes de ordenador pertenece a este grupo, aunque gran parte de las reglas de las gramáticas que los generan pueden reducirse  a las de tipo 2).
Se caracteriza por que la longitud de las formas sentenciales partiendo de S es siempre no decreciente.

\subsection{Gramáticas de tipo 2}
También conocidas como gramáticas independientes del contexto tienen reglas de la forma:
$A\rightarrowv con A\inN\wedgev\in(\Sigma\cupN)^{*}$
Se vuelven a introducir leyes compresoras, pero es fácil demostrar que se puede obtener una gramática equivalente que no las tenga, obteniéndose una definición algo mas restrictiva:
$A\rightarrowv con A \inN\wedge v\in(\Sigma\cupN)^{+}$
además es posible que se tenga la regla $S\rightarrow\varepsilon$
En los lenguajes generados por las gramáticas de este tipo el significado de las "palabras"  es independiente de su posición.
Una última característica de este tipo de gramáticas es que las derivaciones obtenidas al utilizarlas se pueden representar utilizando árboles.
\subsection{Gramáticas de tipo 3}
También conocidas como gramáticas regulares, tienen reglas de la forma:
$A\rightarrowaB \wedge A\rightarrowb o de la formaA\rightarrowBa \wedge A\rightarrowb con A,B\inN \wedgea,b\in\Sigma $

A las gramáticas regulares del primer tipo se las llama gramáticas regulares a derechas, a las del segundo tipo gramáticas regulares a izquierdas, en realidad son totalmente equivalentes. Si $\varepsilon$ pertenece al lenguaje , se tolera la regla $S\rightarrow\varepsilon$

Existe una generalización de este tipo de gramáticas llamadas gramáticas lineales con reglas de la forma:

$A\rightarrowwB \wedge A\rightarrowv o de la forma A\rightarrowBw \wedgeA\rightarrowv con A,B\inN \wedgew,v\in\Sigma^{*}$

son totalmente equivalentes a las gramáticas regulares normales, pero en muchos casos su notación es mas adecuada.

\section{Análisis sintáctico descendente}

La idea es generar una forma sentencial a partir del axioma,reconstruyendo una derivación más a la izquierda en orden inverso. $S\Rightarrow_{l}^{*}w$

\subsection{Conjunto FIRST}

$FIRST_{k}:(\Sigma\cupN)^{*}\rightarrow\Sigma^{i\leqk}$ los primeros k símbolos de una casi forma sentencial.
$FIRST_{1}(\alpha)$: conjunto de terminales que inician las cadenas derivadas de $\alpha$

Propiedades del FIRST:
\begin{itemize}
	\item $FIRST_{k}(aw)=a FIRST_{k-1}(w)  a\in\Sigma   w\in(\Sigma\cupN)^{*}$
	\item $FIRST_{k}(\varepsilon)={\varepsilon}$
	\item $FIRST_{k}(xy) = FIRST_{k}(FIRST_{k}(x)FIRST_{k}(y)) =
	FIRST_{k}(x FIRST_{k}(y))  x,y\in(\Sigma\cupN)^{*} =FIRST_{k}(FIRST_{k}(x)y)$
	\item  Si tengo una regla de producción $A\rightarroww$ 
	$FIRST_{k}(w)\subseteq FIRST_{k}(A)$
\end{itemize}

Calculo del FIRST:
\begin{itemize}
	\item Si X es un terminal $FIRST(X)={X}$.
	\item Si $X\rightarrow\varepsilon$ es una producción$\varepsilon\inFIRST(X)$
	\itemSi $X\rightarrowY_{1},Y_{2}...Y_{n}$ es una producción
	\begin{enumerate}
	\item $FIRST(Y_{1})-{\varepsilon}\subseteq FIRST(X)$
	\item $Si \varepsilon\inFIRST(Y_{k})\forallk<i FIRST(Y_{i})-{\varepsilon}\subseteq FIRST(X) $
	\item $Si \varepsilon\inFIRST(Y_{k})\foralli\leqn \varepsilon \in FIRST(X) $
	\end{enumerate}
\end{itemize}

\subsection{Conjunto FOLLOW}
$FOLLOW_{k}:N\rightarrow\Sigma^{i\leqk}$ símbolos que siguen  a A en las diferentes formas sentenciales.
$FOLLOW_{k}(A)={x:S\Rightarrow^{*}wAy \wedge x\inFIRST_{k}(y)}$
$FOLLOW_{1}(A):$ conjunto de terminales a que pueden aparecer inmediatamente a la derecha del no terminal A en alguna forma sentencial,es decir, el conjunto de terminales a tal que haya una derivación de la forma $S\Rightarrow^{*}\alphaAa\beta$ para algún $\alpha$ y  $\beta$.

Propiedades del FOLLOW:

Si tengo una regla $A\rightarrowxXy  X\inN  x,y\in(\Sigma\cupN)^{*}$
$FIRTS_{k}(y FOLLOW_{k}(A))\subseteq FOLLOW_{k}(X)$

Calculo del FOLLOW:
\begin{itemize}
	\item Si S es e axioma $$\inFOLLOW(S)$
	\item Si $A\rightarrow\alphaB\beta$ es una producción $FIRST(\beta)-{\varepsilon}\subseteqFOLLOW(B)$
	\itemSi hay una producción $A\rightarrow\alphaB\beta con \varepsilon\inFIRST(\beta)$
	$FOLLOW(A)\subseteqFOLLOW(B)  A,B\inN  \alpha\in(\Sigma\cupN)^{*}  \beta\in(\Sigma\cupN)^{+}$
\end{itemize}

\subsection{Tabla de análisis sintáctico predictivo}

A la vista del tope de pila y del símbolo nos dice que acción llevar a cabo.
Está compuesta por entradas de la forma $M[a,X]=(X\rightarroww) a\inFIRST(w.FOLLOW(X))$
 



\section{Análisis sintáctico ascendente}

La idea es generar una forma sentencial a partir del axioma,reconstruyendo una derivación más a la derecha en orden inverso. $S\Rightarrow_{r}^{*}w$



















\imagen{thompson-vacio}{Autómata para una expresión vacía}



\section{Listas de items}

Existen tres posibilidades:

\begin{itemize}
	\item primer item.
	\item segundo item.
\end{itemize}

\begin{enumerate}
	\item primer item.
	\item segundo item.
\end{enumerate}

\begin{description}
	\item[Primer item] más información sobre el primer item.
	\item[Segundo item] más información sobre el segundo item.
\end{description}
	


\section{Tablas}

Igualmente se pueden usar los comandos específicos de \LaTeX o bien usar alguno de los comandos de la plantilla.

\tablaSmall{Herramientas y tecnologías utilizadas en cada parte del proyecto}{l c c c c}{herramientasportipodeuso}
{ \multicolumn{1}{l}{Herramientas} & App AngularJS & API REST & BD & Memoria \\}{ 
HTML5 & X & & &\\
CSS3 & X & & &\\
BOOTSTRAP & X & & &\\
JavaScript & X & & &\\
AngularJS & X & & &\\
Bower & X & & &\\
PHP & & X & &\\
Karma + Jasmine & X & & &\\

} 

\capitulo{4}{Técnicas y herramientas}

\section{Técnicas}

\subsection{SCRUM}
SCRUM\cite{scrum} es una metodología ágil en la que se aplican de manera regular un conjunto de buenas prácticas para trabajar colaborativamente y obtener el mejor resultado posible de un proyecto.

Se basa en la creación de procesos iterativos e incrementales, así se pueden realizar estimaciones de rendimiento futuro sobre las cuales se pueden tomar decisiones y controlar los riesgos.
\subsection{UML\cite{UML}}
Lenguaje Unificado de Modelado (Unified Modeling Language)\footnote{\url{http://www.uml.org/}} es el lenguaje de modelado de sistemas de software más conocido y utilizado en la actualidad en el diseño de aplicaciones Orientadas a Objetos.

Es un lenguaje gráfico para visualizar, especificar, construir y documentar un sistema.

Es importante remarcar que UML es un lenguaje de modelado para especificar o para describir métodos o procesos. Se utiliza para definir un sistema, para detallar los artefactos en el sistema y para documentar y construir. En otras palabras, es el lenguaje en el que está descrito el modelo.
\section{Herramientas}

Todas las herramientas utilizadas para la elaboración del proyecto son gratuitas o se pueden obtener versiones de prueba en Internet.

\subsection{\textit{Java}}
\textit{Java}\footnote{\url{https://www.java.com.es}} es un lenguaje de programación de propósito general, concurrente y orientado a objetos.

Se trata de un lenguaje fuerte y estáticamente tipado.

Distingue entre errores en tiempo de compilación y errores en tiempo de ejecución.

\textit{Java} es un lenguaje de alto nivel, en el sentido de que los detalles de representación no son accesibles al programador. 
Incluye una administración automática del almacenamiento de datos en forma del recolector de basura, diseñado para evitar los problemas relacionados con la liberación manual de memoria. Es uno de los lenguajes de programación mas usados en el mundo.\cite{java}

\textit{Java SE 8} es la edición más reciente del lenguaje de programación \textit{Java}, y representa la mayor evolución del mismo en su historia.

Incluye nuevas características, mejoras y correcciones de bugs para mejorar la eficacia en el desarrollo y la ejecución de programas \textit{Java}. 
 
Las librerías de la plataforma \textit{Java} mantienen una evolución paralela a la del lenguaje.

\subsection{\textit{Eclipse}}
\textit{Eclipse}\footnote{\url{https://eclipse.org/}} es una plataforma de software compuesto por un conjunto de herramientas de programación de código abierto multiplataforma para el desarrollo integrado (\textit{IDE}) de aplicaciones \textit{Java}  y una base para productos basados en \textit{Eclipse Platform}.


\subsection{\textit{JUnit 4}}
\textit{JUnit}\footnote{\url{http://junit.org/}} es un sistema software utilizado para realizar pruebas sobre código \textit{Java}, formando parte de la familia de herramientas de pruebas \textit{xUnit}.
Es una de las librerías \textit{Java} más utilizadas en proyectos de código abierto.

La versión 4 de \textit{JUnit} extiende y simplifica la funcionalidad de anteriores versiones, haciendo un uso extensivo del sistema de anotaciones de \textit{Java}.


\subsection{\textit{Apache Maven}}
\textit{Maven}\footnote{\url{http://maven.apache.org/}} es una herramienta de administración y construcción de proyectos usada principalmente con \textit{Java}.

Su funcionamiento se basa en la existencia de un fichero de configuración XML, el \textit{POM} (Project Object Model).
Este fichero define la construcción, emisión, documentación, empaquetado, pruebas, manejo de dependencias y múltiples otras tareas de manera centralizada.

\subsection{\textit{JavaCC}}
\textit{JavaCC}\footnote{\url{https://javacc.java.net/}} (\emph{Java Compiler Compiler}) es un generador de analizadores sintácticos \textit{Java}.
Funciona mediante la especificación de una gramática en un formato propio.
A partir de este fichero, la herramienta genera un programa Java capaz de procesar texto y reconocer coincidencias con la gramática.

Además del generador en sí, \textit{JavaCC} proporciona una serie de herramientas relacionadas, como un constructor de árboles (\textit{JJTree}) y un informe en caso de error de los mejores entre los analizadores sintácticos disponibles.

\subsection{\textit{VersionOne}}
\textit{VersionOne}\footnote{\url{https://www.versionone.com/}} es un sistema de control de versiones, diseñado para trabajar con proyectos de cualquier tamaño.
\textit{VersionOne} permite mantener múltiples ramas locales independientes, cuya creación, modificación y combinado resulta poco costoso.
Esto permite aislar tareas, trabajando e incorporando cada una de manera totalmente separada.

\subsection{\textit{Github}}
\textit{Github}\footnote{\url{https://github.com/}} proporciona un servicio de almacenamiento de repositorios remotos y un entorno de colaboración para desarrolladores.

\textit{Github} proporciona herramientas de seguimiento de proyectos, incluyendo una wiki y un sistema de seguimiento de problemas (\emph{issue tracker}) por repositorio.
También es compatible con otras aplicaciones web, como por ejemplo \textit{Pivotal Tracker}, que facilitan su integración en el proceso de desarrollo.

Enlace al repositorio de este proyecto:  \url{https://github.com/vrt0004/Trabajo}

\subsection{\textit{Moodle}}
\textit{Moodle} (\emph{Modular Object-Oriented Dynamic Learning Environment}) es una plataforma de enseñanza virtual (\emph{e-learning}) desarrollada como software libre.

Es una aplicación para crear y gestionar plataformas educativas, espacios donde un centro educativo, institución o empresa, gestiona recursos proporcionados por sus docentes, organiza el acceso a los mismos y permite la comunicación entre todos los implicados (alumnado y profesorado).

Una de las características de \textit{Moodle} es su capacidad para importar cuestionarios a partir de varios formatos, incluyendo texto plano, el formato propietario \textit{Gift} y a partir de documentos XML.

\subsection{\LaTeX{}}
\LaTeX{}\footnote{\url{https://latex-project.org/}} es un lenguaje de marcadores para la preparación de documentos comúnmente usado en publicaciones técnicas o científicas.

Es un procesador de texto «\emph{What you see is not what you get}», lo cual significa que lo que vemos durante la edición no es el documento final, sino las instrucciones para generarlo.
Esto permite separar casi completamente el contenido del documento de su formato. La salida obtenida será la misma con independencia del dispositivo o sistema operativo empleado para su visualización o impresión.

\subsection{\TeX{}Maker}
\TeX{}Maker\footnote{\url{http://www.xm1math.net/texmaker/}} es un editor libre de /\LaTeX{}, moderno y multiplataforma para Windows, Linux y MacOSX que integra las herramientas necesarias para desarrollar documentos con \LaTeX{}, en una sola aplicación.

\TeX{}Maker incluye soporte \textit{Unicode}, la corrección ortográfica, auto-completado, plegado de código y un visor de pdf con el apoyo \textit{SyncTeX} y el modo de visualización continua. \TeX{}Maker es fácil de usar y configurar. Se distribuye bajo la licencia \textit{GPL}.

\subsection{MiK\TeX{}}
MiK\TeX{} es una distribución \TeX{}/\LaTeX{} libre de código abierto para \textit{Windows}.
Una de sus características es la capacidad que tiene para instalar paquetes automáticamente sin necesidad de intervención del usuario. Al contrario que otras distribuciones, su instalación es muy sencilla.

\subsection{\textit{Mustache}}
\textit{Mustache}\footnote{\url{http://mustache.github.io/mustache.1.html}} es un sistema de plantillas de lógica descendente para HTML, XML y otros muchos.

Se llama de lógica descendente porque no tiene comandos if, else, ni bucles for. En cambio, sólo se trabaja con etiquetas.
Algunas etiquetas son reemplazadas por un valor, por un conjunto de valores o por nada.

\subsection{\textit{XML}}
\emph{XML} (\emph{Extensive Markup Language}) es un lenguaje de etiquetas que se utiliza para crear documentos estructurados, compuestos de entidades que contienen en su interior datos u otras entidades.
El estándar fue producido y es desarrollado por el \emph{World Wide Web Consortium}\footnote{\url{http://www.w3.org/}}.

Podemos verificar que un documento \emph{XML} tiene el formato correcto validándolo contra un lenguaje de definición de esquemas: \textit{DTD}\footnote{\url{http://es.wikipedia.org}},(\emph{Document Type Definition}), \emph{XML Schema}\footnote{\url{http://www.w3.org/TR/xmlschema-0/}}, \textit{RELAX NG}\footnote{\url{http://www.relaxng.org/}}, etc.

También podemos verificar que un documento está <<bien formado>>, es decir, que cumple una serie de reglas gramaticales mínimas.

\subsection{\textit{Commons CLI}}
La biblioteca \textit{Apache Commons CLI}\footnote{\url{http://commons.apache.org/proper/commons-cli/}} proporciona una API para analizar las opciones de línea de comandos pasados a los programas.

También es capaz de imprimir mensajes de ayuda que detallan las opciones disponibles para una herramienta de línea de comandos.

\subsection{\textit{ObjectAid UML Explorer}}
\textit{ObjectAid UML Explorer}\footnote{\url{http://www.objectaid.com/}} es un complemento de visualización de código para \textit{Eclipse}.
Permite mostrar el código fuente de un proyecto Java en forma de diagramas \textit{UML}, reflejando el estado y las relaciones en el código, y actualizándose a medida que el código cambia.


\subsection{\textit{EclEmma}}
\textit{EclEmma}\footnote{\url{http://www.eclemma.org/}} es una herramienta que permite examinar la cobertura de pruebas en un proyecto \textit{Java}.
Dispone un un complemento para \textit{Eclipse} que permite realizar las comprobaciones directamente desde el \textit{IDE}.

\subsection{\textit{Jabref}}
\textit{Jabref}\footnote{\url{http://www.jabref.org/}} es un editor de referencias bibliográficas que permite introducir las referencias bibliográficas en \TeX{}Maker de forma sencilla.

\subsection{\textit{SourceMonitor}}
\textit{SourceMonitor}\footnote{\url{http://www.campwoodsw.com/sourcemonitor.html}} es una herramienta que permite obtener las métricas del codigo de la aplicación \textit{PLGRAM}.

\subsection{\textit{DIA}}
\textit{DIA}\footnote{\url{https://sourceforge.net/projects/dia-installer/}} es una herramienta para dibujar diagramas de estructuras. \textit{Dia Diagram Editor} es un software gratuito de dibujo de código abierto.





\capitulo{5}{Aspectos relevantes del desarrollo del proyecto}
En este proyecto se presenta una aplicación de escritorio que permite generar cuestionarios sobre análisis ascendente y descendente. Además dicha aplicación permite su posterior publicación en Moodle y en formato \LaTeX{} para su futuro uso en pruebas escritas.


\section{Conocimientos necesarios}
Para realizar esta aplicación ha sido necesario repasar y utilizar algunos de los conocimientos adquiridos durante estos últimos años en el grado.


\begin{itemize}
\item Estructuras de datos.

Uno de los problemas más importantes durante el desarrollo fue la búsqueda de una estructura de datos adecuada para usar en las plantillas. Los conocimientos en estructuras de datos han servido para obtener un software más eficiente y sencillo.
\item Ingeniería de software.

Ha sido utilizada para dar un enfoque sistemático, disciplinado y cuantificable a esta aplicación.

\item Gestión de proyectos.

Se ha seguido una metodología \textit{Scrum}. Ha sido necesario repasar los apuntes para seguirla de forma correcta.

\item Procesadores de lenguajes.

Ha sido necesario repasar toda la teoría de dicha asignatura para comprobar que los resultados obtenidos de la aplicación eran correctos.

\end{itemize}  
\section{Conocimientos adquiridos}
Se han adquirido muchos conocimientos durante el desarrollo de \textit{PLGRAM}.
\begin{itemize}
\item Uso de plantillas.

Para generar los documentos se ha utilizado un motor de  plantillas llamado \textit{Mustache}.
\item Archivos .XML.

Los documentos generados con la aplicación \textit{PLGRAM} pueden ser documentos .XML, por lo que se han adquirido conocimientos sobre su diseño y funcionamiento. 
\item Archivos .TEX.

Los documentos generados con la aplicación \textit{PLGRAM} también pueden ser documentos .TEX, por lo que se han adquirido conocimientos sobre su diseño y funcionamiento. 


\item Uso de \TeX{}MAKER

Realizar la documentación del trabajo de fin de grado con \LaTeX{} ha sido un reto que me será útil para el futuro laboral.


\item Uso de \textit{Pluggins} en Eclipse

La utilización de diferentes pluggins en Eclipse ha facilitado la realización de la aplicación y que el resultado sea una aplicación de calidad.
\end{itemize}

\section{Objetivos realizados}
El objetivo ha sido realizar  una aplicación con una finalidad docente, que facilite la construcción de cuestionarios para poder utilizar como herramienta de aprendizaje y de evaluación en la plataforma Moodle y que también  obtenera de forma sencilla documentos en formato TEX que pueden permitir, por ejemplo, realizar pruebas escritas.
\capitulo{6}{Trabajos relacionados}

Para comenzar a trabajar en el desarrollo de la presente aplicación se buscó información relacionada con la aplicación que se quería llevar a cabo.

Para ello, se realizó una investigación a través de la Web sobre  aplicaciones que tuvieran una función similar a la desarrollada. 

Aunque no existe una gran variedad de programas que realicen tareas similares a $PLGRAM$, entre los dos programas recomendados por el tutor se  ha encontrado información suficiente para poder llevar a cabo la aplicación. Dichos programas son $PLQUIZ$\cite{PLQUIZ} y $BURGRAM$\cite{BURGRAM}.

Dado que ambos programas han sido desarrollados por alumnos de la Universidad de Burgos, como trabajos de final de grado, el tutor pudo facilitarme su estructura y toda la documentación que estimó necesaria para la correcta ejecución de $ PLGRAM $.


\section{PLQUIZ \cite{PLQUIZ}}

PLQUIZ\footnote{Roberto Izquierdo Amo, Julio 2014, Universidad de Burgos} es una herramienta de escritorio que permite generar preguntas de test aleatorias (tipo quiz, cloze, de texto libre...) sobre problemas de algoritmos de análisis léxico. El formato utilizado para generar las preguntas es .XML para importarse a entornos virtuales de aprendizaje (Moodle), y .TEX para  su impresión en papel a través de la obtención de código \LaTeX{}.

Se han seguido sus pasos en el formato de la interfaz gráfica, pues tiene una organización muy clara para el usuario, permitiendo acceder rápidamente a las opciones disponibles.

\imagen{PLQUIZ}{PLQUIZ}


\section{BURGRAM \cite{BURGRAM}}

BURGRAM\footnote{Carlos Gómez Palacios, Enero 2008, Universidad de Burgos} es una herramienta que permite seguir paso a paso el proceso de los distintos modos de análisis sintáctico (tanto ascendente como descendente). La aplicación permite editar gramáticas, visualizar el proceso completo de operación del algoritmo sobre las mismas y la generación de informes con los resultados.

Se han reutilizado varias de las clases de este programa pues realizan el análisis que la aplicación desarrollada necesitaba para obtener los datos que posteriormente se han tratado para obtener los ficheros .XML.
\imagen{BURGRAM}{BURGRAM}


\section{SEFALAS \cite{SEFALAS}}
SEFALAS\footnote{José Francisco Jódar Reyes, 2004, Universidad de Granada} (\textbf{S}oftware para la \textbf{E}nseñanza de las \textbf{F}ases de \textbf{Á}nálisis \textbf{L}éxico y \textbf{A}nálisis \textbf{S}intáctico) es una aplicación desarrollada como herramienta para la enseñanza de ciertas técnicas de análisis léxico y de análisis sintáctico. En lo referente a análisis sintáctico SEFALAS recoge las dos estrategias: análisis descendente y análisis ascendente. La primera de ellas se ilustra mediante el método descendente LL(1). La segunda se ilustra con los métodos de precedencia de operador y precedencia simple y las gramáticas LR en sus diversas modalidades: SLR, LR(1) y LALR. Además, es posible escribir la gramática en formato YACC y Sefalas construye la tabla de análisis dependiendo de la estrategia seleccionada. Es posible realizar el análisis de un texto de entrada y ver la evolución del análisis de forma interactiva a través de la tabla de análisis.
\imagen{SEFALAS}{SEFALAS}


\section{ANAGRA \cite{anagra}}

ANAGRA\footnote{Raúl Novoa Mínguez,2009,Universidad de Zaragoza} es una aplicación que permite editar y abrir gramáticas escritas con la sintáxis Yacc.
Calcular las funciones del FIRST y FOLLOW, y otras operaciones con gramáticas.
\imagen{ANAGRA}{ANAGRA}

\section{Proyecto SEPa \cite{SEPa}}

SEPa\footnote{Juan José Tamagnini,Salvador Valerio Cavadini y
Pablo Luis Berdaguer,2003 ,Universidad católica de Santiago del Estero} es una aplicación que tiene como objetivo facilitar al estudiante el aprendizaje autónomo de conceptos y procedimientos que realizan los compiladores.
\imagen{sepa}{SEPa}


\capitulo{7}{Conclusiones y Líneas de trabajo futuras}
\section{Conclusiones}

Se ha perfeccionado el conocimiento sobre Java y su integración con bibliotecas de terceros como ha sido \textit{Commons CLI}.

El empleo de una herramienta para el control de versiones, ya utilizados con anterioridad,
ha servido para reforzar la idea de lo imprescindibles que son estas herramientas para el
desarrollo software y las garantías de seguridad que ofrecen.

Se ha aprendido un nuevo modo de realizar documentos técnicos con el uso de \LaTeX{}. Aunque
en un principio supuso una carga a la documentación, a medida que avanzó el desarrollo, favoreció el interés por la escritura debido a los retos que se presentaron durante su ejecución.

Por último, destacar el perfeccionamiento de todas las tareas del desarrollo software aprendidas a lo largo del grado en ingeniería: planificación,
análisis, diseño, implementación, pruebas y documentación.
\section{Líneas de trabajo futuras}

\subsection{Deshacer cambios}
La aplicación no permite \textit{volver atrás} si se elimina parte del trabajo realizado. Por ejemplo, si se elimina una cuestión, no se puede recuperar.

Desarrollar una opción de <<deshacer>> evitaría problemas al usuario y aumentaría la usabilidad del programa.

\subsection{Exportación de mas de una pregunta en la interfaz gráfica}
La aplicación exporta las preguntas en formato .XML o .TEX de una en una.

Una posible mejora de la aplicación consistiría en poder exportar todas las preguntas de una sola vez, facilitando el trabajo y reduciendo el tiempo de uso.

\subsection{Más tipos de cuestiones}
 La aplicación proporciona un conjunto relativamente amplio de modelos de preguntas, pero, partiendo de la funcionalidad de resolución de algoritmos ya implementada, podrían generarse nuevos tipos de cuestiones para evaluar conocimientos de manera más amplia. 

Un ejemplo serían preguntas de emparejamiento, que están directamente soportadas por \textit{Moodle} y son sencillas de crear en otros formatos

  
 \subsection{Importar cuestionarios} 
En la versión actual de la aplicación, los cuestionarios pueden exportarse pero no importarse. Esto quiere decir que, una vez se cierra el programa, no se puede continuar trabajando sobre las preguntas que se tenían, a menos que se cree un cuestionario nuevo y se añadan las preguntas a mano. 

Se proponen dos posibles implementaciones de un sistema de almacenamiento y carga:

 \begin{itemize}
 \item El almacenamiento de los cuestionarios como ficheros separados, ya sea en texto plano, XML o cualquier otro formato.
 \item Consistiría en la lectura de un fichero exportado y su comparación con la plantilla de la que se generó. De esta manera se pueden localizar la gramática y el análisis y extraer la plantilla. Una vez obtenidas, la reconstrucción del cuestionario es trivial.
 \end{itemize}



\bibliographystyle{plain}
\bibliography{bibliografia}

\end{document}